\documentclass{article}
\usepackage[T1, T2A]{fontenc}
\usepackage[russian, english]{babel}
\usepackage{indentfirst}

\begin{document}
\begin{titlepage}
    \begin{center}
        \normalsize
        МГТУ им Н.Э. Баумана, кафедра ИУ5 \\
        \vspace*{1cm}
        \LARGE
        \textbf{Лабораторная работа №2 по дисцилине "РИП"}

        \vspace{0.5cm}
        Вариант 19
    \end{center}
    \vfill

    \begin{flushright}
        \textbf{Выполнил:} Никольский Даниил, ИУ5-51б \\
        \textbf{Проверил: } Гапанюк Юрий Евгеньевич, ИУ5 \\
    \end{flushright}
    \vspace{1.5cm}
    \begin{flushleft}
        \textbf{Дата: \today} \\
        \textbf{Подпись: Никольский Д.Р.} \\
    \end{flushleft}
\end{titlepage}

\tableofcontents
\newpage

\section{Задание}
\subsection{Задача 1}
Необходимо реализовать генераторы field и gen\_random. Генератор field последовательно выдает значения ключей словарей массива

\begin{enumerate}
    \item В качестве первого аргумента генератор принимает list, дальше через *args генератор принимает неограниченное кол-во аргументов.

    \item Если передан один аргумент, генератор последовательно выдает только значения полей, если поле равно None, то элемент пропускается

    \item Если передано несколько аргументов, то последовательно выдаются словари, если поле равно None, то оно пропускается, если все поля None, то пропускается целиком весь элемент

\end{enumerate}

Генератор gen\_random последовательно выдает заданное количество случайных чисел в заданном диапазоне

Пример: \\
gen\_random(1, 3, 5)должен выдать 5 чисел от 1 до 3, т.е. примерно 2, 2, 3, 2, 1

В ex\_1.py нужно вывести на экран то, что они выдают, с помощью кода в одну строку. Генераторы должны располагаться в librip/gen.py

\subsection{Задача 2}
Необходимо реализовать итератор, который принимает на вход массив или генератор и итерируется по элементам, пропуская дубликаты. Конструктор итератора также принимает на вход именной bool-параметр ignore\_case, в зависимости от значения которого будут считаться одинаковыми строки в разном регистре. По умолчанию этот параметр равен False. Итератор не должен модифицировать возвращаемые значения.

В ex\_2.py нужно вывести на экран то, что они выдают одной строкой. Важно продемонстрировать работу как с массивами, так и с генераторами (gen\_random).
Итератор должен располагаться в librip/iterators.py

\subsection{Задача 3}
Дан массив с положительными и отрицательными числами. Необходимо одной строкой вывести на экран массив, отсортированный по модулю. Сортировку осуществлять с помощью функции sorted

\subsection{Задача 4}
Необходимо реализовать декоратор print\_result, который выводит на экран результат выполнения функции. Файл ex\_4.py не нужно изменять.
Декоратор должен принимать на вход функцию, вызывать её, печатать в консоль имя функции, печатать результат и возвращать значение. 
Если функция вернула список (list), то значения должны выводиться в столбик.
Если функция вернула словарь (dict), то ключи и значения должны выводить в столбик через знак равно

\subsection{Задача 5}
Необходимо написать контекстный менеджер, который считает время работы блока и выводит его на экран

\subsection{Задача 6}
Мы написали все инструменты для работы с данными. Применим их на реальном примере, который мог возникнуть в жизни. В репозитории находится файл data\_light.json. Он содержит облегченный список вакансий в России в формате json.
Структура данных представляет собой массив словарей с множеством полей: название работы, место, уровень зарплаты и т.д.
В ex\_6.py дано 4 функции. В конце каждая функция вызывается, принимая на вход результат работы предыдущей. За счет декоратора @print\_result печатается результат, а контекстный менеджер timer выводит время работы цепочки функций.
Задача реализовать все 4 функции по заданию, ничего не изменяя в файле-шаблоне. Функции f1-f3 должны быть реализованы в 1 строку, функция f4 может состоять максимум из 3 строк.

\begin{enumerate}
    \item Функция f1 должна вывести отсортированный список профессий без повторений (строки в разном регистре считать равными). Сортировка должна игнорировать регистр. Используйте наработки из предыдущих заданий.
    \item Функция f2 должна фильтровать входной массив и возвращать только те элементы, которые начинаются со слова “программист”. Иными словами нужно получить все специальности, связанные с программированием. Для фильтрации используйте функцию filter.
    \item Функция f3 должна модифицировать каждый элемент массива, добавив строку “с опытом Python” (все программисты должны быть знакомы с Python). Для модификации используйте функцию map.
    \item Функция f4 должна сгенерировать для каждой специальности зарплату от 100 000 до 200 000 рублей и присоединить её к названию специальности. Используйте zip для обработки пары специальность — зарплата.
\end{enumerate}


\end{document}